Le language de programmation C\# semble est le plus adapté pour le développement de \textit{\gameName}. 
Dérivée du C++ et étonnamment proche du \textit{Java}, Il permet de coder des jeux vidéos grâce à sa compatibilité avec les moteurs de jeu \textit{Godot} ou encore \textit{Unity}.
Nous utiliserons l’environnement de développement intégré (IDE) \textit{Rider} proposé par \textit{JetBrains}, qui  nous permettra d’avoir à disposition tous les outils nécessaires pour coder notre projet du début à la fin.
\\

Ensuite, pour les textures graphiques du jeu, nous utiliserons le logiciel de retouche \textit{Adobe Photoshop}, qui nous permettra donc de traiter les différents environnements, les différentes textures au sein du jeu et tout ce qui touche les graphismes en général.
Ce sera sans doute le logiciel principal du directeur graphique du groupe, c'est un outil très complet et très puissant qui répondra à tous ses besoins.
\\

Puis, nous utiliserons principalement \textit{GitHub} comme environnement de travail, qui est est un service web d’hébergement et de gestion de développement de logiciels, utilisant le logiciel de gestion de versions \textit{Git}. 
L’utilisation de \textit{GitHub} nous permettra de mettre en commun nos tâches et de les réunir dans notre projet global, en guise d’exemple, un membre du groupe qui doit fournir un code pour un autre membre peut directement passer par \textit{GitHub} pour lui fournir, ou encore deux membres qui ont la même tâche peuvent travailler et modifier le code en même temps grâce au dépot \textit{Git}, et ainsi travailler plus efficacement.
\\

En plus de cela, notre groupe utilisera le logiciel de messagerie instantanée \textit{Discord} pour essentiellement communiquer sur l’avancée du projet, mais aussi partager l’avancement de chacun sur ses tâches ou encore s’organiser sur les prochaines tâches du projet. Tout cela peut se résumer en de brèves discussions jusqu’à des réunions en appel vocal qui peuvent durer un certain temps rendant la répartition des différentes tâches efficace.
\\

Nous utiliserons aussi le moteur de jeu multiplateforme \textit{Godot} pour la création de notre jeu vidéo.
Il contient tout ce dont on aura besoin pour le développement de \textit{\gameName}, comme le moteur 2D, le moteur physique, un gestionnaire d’animations ou encore des langages de script pour la programmation des comportements. 
Comme cité précédemment, c’est avec le langage C\# que l’on utilisera le moteur \textit{Godot}, afin de mener à bien notre projet.
\\

Enfin, au niveau du matériel, nous utiliserons chacun nos pc personnels pour avancer sur les tâches et les projets depuis chez soi, mais nous aurons également à disposition et utiliserons par ailleurs les pc des locaux de la \textit{Stonks Industries}, situés en salles machines de l’EPITA, qui nous permettra de travailler directement sur place avec un environnement purement professionnel.
\\
