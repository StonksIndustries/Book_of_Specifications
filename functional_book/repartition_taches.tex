% Répartition : le cahier des charges doit contenir un planning détaillé de répartition des tâches par personne. 
% Vous le présenterez sous la forme d’un tableau à deux entrées (tâches, personnes) avec deux personnes par tâche : le responsable et son suppléant.

% Corrigé

L'attribution des tâches est primordiale pour le bon déroulement du projet.
Chaque tâche est attribuée à un membre de l'équipe et il en est le responsable.
C'est son rôle de veiller à ce que la tâche soit bien réalisée dans les temps et qu'elle réponde aux besoins pour laquelle elle a été créée.
Le responsable peut diviser sa tâche en sous-tâches et les attribuer à d'autres membres de l'équipe.
Cela lui permet de se concentrer sur sa tâche principale et d'arriver à un résultat plus rapidement.
\\

On assigne en plus un suppléant à celle-ci, qui pourra prendre le relais en cas d'indisponibilité du responsable.
En effet une tâche ne peut pas être laissée à l'abandon, elle pourrait mettre en retard tout le projet.
Ces rôles sont choisis en fonction des compétences de chacun, mais aussi en fonction de leurs disponibilités dans le projet.
Il est important de cerner les capacités des membres du groupe afin que leurs compétences soit utilisées au mieux.
\\

Les tâches sur le long terme (d'une durée supérieure à une semaine de travail)
sont listées dans le tableau ci-dessous (cf Figure \ref*{fig:repartition_des_taches}).
Il permet de voir qui est responsable de quelle tâche et de s'assurer que chaque tâche est bien assign\'ee à quelqu'un.
Ce tableau réunit les tâches par thèmes pour plus de lisibilité.

\begin{figure}[H]
    \fontsize{10pt}{15pt}\selectfont
    \centering
    \begin{tabular}{|c|c|c|c|}
        \hline
        \bfseries{Cat\'egorie} & \bfseries{T\^ache} & \bfseries{Responsable} & \bfseries{Suppl\'eant} \\
        \hline\hline
            % Général
            G\'en\'eral & Cahier des charges & Martin & Mohamed Aziz \\
        \hline\hline
            % Conceptualisation du jeu
            \multirow{2}{*}{Conceptualisation du jeu} & \'Ecriture du sc\'enario & Alexandre & Ayemane \\
            \cline{2-4}
            & \'Ecriture des dialogues & Ayemane  & Mohamed Aziz \\
        \hline\hline
            % Site Web
            \multirow{3}{*}{Site Web} & Design du site web & Ayemane & Martin \\
            \cline{2-4}
            & Recherche technologie web & Martin & Michaël \\
            \cline{2-4}
            & Development du site web & Martin & Michaël \\
        \hline\hline
            % Texturing
            \multirow{2}{*}{Texturing} & Design des interfaces (UI) & Mohamed Aziz & Ayemane \\
            \cline{2-4}
            & Design personnages / terrains / objet & Ayemane & Alexandre \\
        \hline\hline
            % Intégration des textures
            \multirow{2}{*}{Intégration des Textures} & Int\'egration des interfaces dans Godot & Mohamed Aziz & Martin \\
            \cline{2-4}
            & Int\'egration des textures dans Godot & Mohamed Aziz & Ayemane \\
            \hline\hline
            % Level design
            \multirow{2}{*}{Level design} & Faire un schema de la carte & Alexandre & Mohamed Aziz \\
            \cline{2-4}
            &Level mapping avec les textures & Alexandre & Ayemane \\
        \hline\hline
        % Système d'équipements
            \multirow{3}{*}{Système d'équipements} & Listing items et objets (al\'eatoire) & Alexandre & Martin \\
            \cline{2-4}
            & Gestion aléatoire des objets & Michaël & Alexandre \\
            \cline{2-4}
            & Equilibrage des objets & Alexandre & Martin \\
        \hline\hline
            % Multijoueur
            & Multijoueur & Mohamed Aziz & Michaël \\
            \hline\hline
            % IA
            \multirow{2}{*}{IA} & Combat des mobs & Michaël & Martin \\
            \cline{2-4}
            & Errance des mobs & Michaël & Martin \\
        \hline\hline
            % Musique et effet sonores
            & Musique et effet sonores & Mohamed Aziz & Martin \\
        \hline
    \end{tabular}
    \caption{R\'epartition des tâches}
    \label{fig:repartition_des_taches}
\end{figure}


Ce tableau présente les tâches sur le long terme, mais il est possible que des tâches plus courtes soient assignées à des membres de l'équipe.
Pour cela, nous avons fait le choix d'utiliser les outils proposés par GitHub.
En effet, les \textit{issues} permettent de lister les problèmes et les tâches à effectuer directement sur le dépôt du projet.
Cette intégration au plus proche du code permet de faciliter la gestion des tâches et la communication entre les membres de l'équipe.
\\






% Conceptualisation du jeu
    % Écriture du scenario
    % Écriture des dialogues

% Programmation
    % Proof of Concept / Test en Godot
    % Implémentation / Développement
    % Review / Debug / Publishing

% Site web
    % Design du site
    % Recherche technologie web
    % Développement du site
    
% Texturing
    % Réflexion sur le design globale (dessin)
    % Design des interfaces (UI)
    % Design personnages
    % Design terrains
    % Design objets

% Intégration des textures
    % Intégration des interfaces
    % Intégration des textures personnages
    % Intégration des textures terrains
    % Intégration des textures objets

% Level design
    % Faire un schema de la carte 
    % Level mapping avec les textures

% Système d'équipements
    % Listing items et objets (aléatoire)
    % Boutique (aléatoire) -> PNG et inventaire
    % Armurerie (aléatoire) -> Forgeron

% Mutlijoueur

% IA

% Musique et effet sonores


