% Corrigé


L'équipe de développement du projet est composée de 5 membres, tous étudiants à l'EPITA en classe de B1.
Chaque membres de l'équipe possède des compétences et des connaissances dans des domaines différents.
Cette polyvalence permet à l'équipe de répondre à tous les besoins du projet et de travailler de manière efficace et autonome.


\subsubsection*{Mohamed Aziz Ben Amor}

Aziz Ben Amor, âgé de 18 ans, est un élève dévoué à l'EPITA. 
Après avoir choisi des spécialités scientifiques axées sur l'informatique, il s'est inscrit à l'EPITA, une école renommée dans le secteur de l'informatique. 
Passionné par son domaine, il se consacre entièrement à la réussite de son parcours académique.
Cependant, ses intérêts ne se limitent pas à la programmation et au développement ; il est également fasciné par le monde de la finance et de l'entrepreneuriat. 
Il a acquis des connaissances en gestion d'entreprise, grâce à son intérêt pour ces sujets. 
Étant d'origine tunisienne, il a néanmoins été éduqué dans le système éducatif français, il maîtrise l'arabe maghrébin et l'anglais. 


% \newpage

\subsubsection*{Ayemane Bouarbi}

En tant que lead game artist, Ayemane, étudiant de génie à EPITA, à 18 ans seulement, se voit confier la lourde responsabilité de piloter la direction artistique du projet de jeu vidéo ambitieux \textit{\gameName}. 
Et ce choix n’a pas été realisé sans raison: Ayemane est un artiste du jeu vidéo ayant contruit sa propre réputation avec ses plus de 10 années d'expérience dans l’industrie. 
En effet, passionné par les jeux vidéo et de dessin depuis l’enfance, il commence à créer des jeux vidéo sur scratch dès l'âge de 7 ans avant de commencer à en faire sur python, et plus tard, en C++. 
Il a également de l’expérience dans la création de concepts art, la modélisation 3D, le texturing, le shading et le lighting.
Dans son parcours professionnel, il a participé à la direction artistique de plusieurs jeux vidéos, que cela soit avec de petits studios indépendants ou bien comme sur certains jeux classés AAA.
Dans son rôle actuel de Lead Game Artist chez les Studios \textit{\companyName}, il est responsable de la gestion et de la vision artistique du jeu, ainsi que de la  supervision de la création des assets et des textures du jeu.

\subsubsection*{Alexandre Colsch}

Alexandre Colsch a 17 ans et est étudiant en première année à l’EPITA.
Il est entre autres le cerveau derrière l’univers et l’histoire qui entourent le jeu.
Dès son plus jeune âge, il invente toutes sortes d’histoires solides et détaillées, et depuis cette base solide de création d’histoires, 
Alexandre va mettre à profit son talent au service de \textit{Stonks Industries} pour permettre aux jeux de l’entreprise d’avoir tout un univers très solide et un scénario digne des plus grands films Hollywoodiens.
Il a intégré l’EPITA pour mettre à profit sa passion pour l’informatique et la développer afin d’obtenir un très bon niveau en la matière.
Alexandre est également passionné par l’automobile et le sport automobile en général, éprouvant un très grand intérêt sur le sujet.
Sa passion se concrétise par ses bonnes performances en karting, sport qu’il adore faire mais il a très peu d'opportunités d’en faire.

Il est également un grand passionné de jeux vidéos, un loisir qu’il aime faire et qu’il n'hésite pas à faire durant son temps libre.
Alexandre a intégré l'équipe de développement de \textit{Stonks Industries} dès sa création et travaille sur le développement du premier jeu de l’entreprise dans le même but que tous les autres membres de l'équipe, créer son propre jeu vidéo.

\subsubsection*{Martin Pasquier}
    
Martin Pasquier est le directeur technique du projet, à l'âge de 17 ans, il a déjà une bonne expérience dans le domaine de la programmation et du développement web.   
En tant que directeur technique, il est responsable de la gestion de l'équipe de développement et de la planification des tâches.
Il est également responsable de la conception et de la mise en œuvre du site web.

Sa passion pour la programmation ayant commencé à un age où le jeu vidéo était très présent dans sa vie.
Il semble évident que Martin a toujours voulu concevoir son propre jeu vidéo.
De par son attrait pour la culture médiévale fantastique, il assistera sans problème le game designer en prenant part à la conception du scenario et de l'histoire du jeu.

Martin est également un grand fan de musique, il aime écouter tous types de musique, ce qui lui sera utile pour la partie sonore du jeu.
La diversité de ses goûts musicaux lui permettra de créer une bande sonore riche et variée, qui s'adaptera parfaitement à l'ambiance du jeu.


\subsubsection*{Michaël Museux}

Michaël Museux, âgé de 17 ans, est étudiant à l'EPITA en première année de classe préparatoire intégrée. 
Depuis un très jeune âge, il est passionné par l'informatique, et plus particulièrement par la programmation.
Après avoir découvert Scratch, il a par exemple écrit des programmes pour résoudre des calculs complexes en mathématique, parfois même impossibles à faire à la main.
Michaël est aussi passionné par l'algorithmie, son excellent niveau dans ce domaine lui a même permis de participer à des concours d'algorithmie, comme le concours Castor ou Algorea où il s'est classé parmi les meilleurs, alors qu'il n'était qu'au lycée. 
Attiré par l'idée de se faire ses propres outils, il s'est aussi intéressé à l’intelligence artificielle (IA) et notamment les technologies d’automatisation.
En effet, à une époque où la popularité des IAs grandissait de plus en plus, il semblait évident pour lui de s'y intéresser, et peut-être même d'en faire son métier.

Par évidence, Michaël intègre l'équipe de développement de \textit{\companyName}, qui commence le développement d'un nouveau jeu : \textit{\gameName}. 
Son rôle dans l'équipe est donc tout trouvé, il sera le développeur principal du jeu, avec un rôle important dans l'intelligence artificielle du jeu.
De plus, ses quelques compétences en développement web lui permettent de participer au développement du site web de l'entreprise.

