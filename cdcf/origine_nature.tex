\vspace*{0.2cm}

Le projet “Lands of Stonkseria” est un jeu vidéo produit par le Studio Stonks Industries.
Il s’agit d’un RPG en 2D, en perspective du dessus.
L’idée de ce projet nous est venue de grands classiques du jeu vidéo tel que les premiers opus de la licence Zelda, ou encore de jeux plus récents comme Stardew Valley, Cult of The Lamb et Don’t Starve Together.
En effet nous avons été séduit par l'expérience immersive que ces jeux ont à nous proposer, que cela soit en matière de jouabilité ou bien d’histoire, parfait pour la création d’un jeu vidéo au contenu riche et intéressant.

C’est donc pour cela que nous nous sommes orientés vers la création d’un RPG en vue du dessus, dans un univers médiéval fantastique, idéal pour l’implémentation d’une histoire intrigante et remplie d’aventures.
Cependant ce projet possèdera également sa propre identité ainsi que des mécaniques de jeu uniques tel que des cycles jour et nuits, la défense de base, ainsi que des assauts d’ennemis qui viendront ajouter du défis à l'expérience du joueur.

\subsubsection*{\hspace*{0.6cm}Description du projet}

Le jeu se déroule dans un monde fantastique où les joueurs incarnent des monstres se défendant contre des humains.
Dès le début du jeu, les joueurs choisiront une race de monstres, chacune avec des capacités et des caractéristiques uniques.
Au fil de l'aventure, ils pourront débloquer de nouvelles capacités et même d'autres races à jouer.
Nous noterons également que “Lands of Stonkseria” sera jouable en multijoueur, parfait pour créer des expériences  inoubliable entre amis .
Les joueurs seront plongés dans une aventure immersive, alternant entre combats, quêtes, dialogues, et cinématiques, le tout en explorant une vaste carte regorgeant de mystères et de défis.

\subsubsection*{\hspace*{0.6cm}Style graphique}

En ce qui concerne le style graphique du projet, nous nous somme orienté vers l’art du pixel étant donné que celui-ci se fond bien dans le style Médiéval fantastique.
De plus, l’art du pixel est un style non seulement intemporel, mais il permet aussi la création relativement peu coûteuse de textures sans pour autant dégrader la qualité visuelle du projet.
Nous noterons également que ce choix artistique peut potentiellement offrir une expérience plus nostalgique aux joueurs les plus anciens, ayant été habitués au graphisme des jeux arcades.
