Dans ce vaste monde qu’est le jeu vidéo, 5 étudiants débarquèrent de nulle part avec en tête une idée ambitieuse, créer de toutes pièces un jeu vidéo. Dans une funeste atmosphère que représente une sombre salle machine de l’EPITA, très tard sous une nuit de pleine lune, fut créée la Stonks Industries. Dans un but commun de devenir une des entreprises les plus prestigieuses dans ce monde, le premier jeu de la Stonks Industries, nommé “Lands of Stonkseria”, vit le jour.

Lands of Stonkseria est avant tout un jeu video de role se déroulant dans le monde médiéval fantastique de Stonkseria, qui est contrôlé par un empire dictatorial. Le joueur aura le choix d’incarner plusieurs classes d’humains dans une faction indépendantiste et aura comme but de combattre l’empire en place et de le renverser, dans un monde où de nombreuses créatures malveillantes errent sans but dans la nuit.

Ce Cahier des Charges présentera en premier lieu le Cahier des Charges Fonctionnel (CdCF), qui se base sur les besoins fonctionnels du projet et la manière dont on va le traiter sur le temps imparti. Il se divise en plusieurs grandes catégories que sont :

L’origine et la nature du jeu Lands of Stonkseria, qui présente donc d'où vient l'idée de la création de ce jeu et qui présente également le type de jeu qu’est Land of Stonkseria


Stonks Industries souhaite développer dans un contexte académique, pour l’EPITA, un nouveau jeu intitulé Lands of Stonkseria, Cette initiative découle de la demande du marché ainsi que de la validation du semestre 2,
    L’objectif principal est de développer un jeu captivant avec des graphismes de haute qualité pour assurer un engagement maximal des joueurs.
    Sont impliqués dans ce projet les équipes de développement, d’art et de scénario, ainsi que les différents enseignants et jurys d’EPITA.
    Ce cahier définit les exigences requises à la conception de ce jeu ainsi que les différentes échéances à venir. Ces exigences comprennent des graphismes avancés, une jouabilité équilibrée sous Windows, le choix d’un moteur de jeu parmi la liste présentée par EPITA. Par ailleurs, nous nous engageons à respecter les droits d’auteur pour les différentes ressources qui vont être utilisées.
    La validation de ce cahier des charges sera assurée par les enseignants d’EPITA, et le suivi de ce jeu sera effectué lors de deux soutenances en 2024.