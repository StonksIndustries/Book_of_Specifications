% Répartition : le cahier des charges doit contenir un planning détaillé de répartition des tâches par personne. 
% Vous le présenterez sous la forme d’un tableau à deux entrées (tâches, personnes) avec deux personnes par tâche : le responsable et son suppléant.


L'attribution des t\^aches est primordiale pour le bon déroulement du projet. 
Chaque t\^ache est attribuée à un membre de l'équipe, il en est le responsable.
C'est son rôle de veiller à ce que la tâche soit bien réalisée dans les temps et qu'elle remplisse le besoin pour lequel elle a été créée.
Le responsable peut diviser sa tâche en sous-tâches, et les attribuer à d'autres membres de l'équipe.
Cela lui permet de se concentrer sur la tâche principale, et d'arriver à un résultat plus rapidement.
\\

On assigne en plus un suppl\'eant à celles-ci, qui pourra prendre le relais en cas d'indisponibilité du responsable. 
En effet une tâche ne peut pas être laissée à l'abandon, elle pourrait mettre en retard tout le projet.
Ces rôles sont choisis en fonction des compétences de chacun, mais aussi en fonction de leurs disponibilités dans le projet.
\\

Les tâches sur le long-terme (d'une durée supérieure à une semaine de travail)
sont list\'ees dans le tableau ci-dessous (cf Figure \ref*{fig:repartition_des_taches}).
Il permet de voir qui est responsable de quelle tâche et de s'assurer que chaque tâche est bien assign\'ee à quelqu'un.
Ce tableau r\'eunit les t\^aches par th\`emes pour plus de lisibilit\'e.


\begin{figure}[H]
    \centering
    \begin{tabular}{|c|c|c|c|}
        \hline
        \bfseries{Cat\'egorie} & \bfseries{T\^ache} & \bfseries{Responsable} & \bfseries{Suppl\'eant} \\
        \hline\hline
        % Général
        G\'en\'eral & Cahier des charges & Martin & Mohamed Aziz \\
        \hline\hline
        % Conceptualisation du jeu
        \multirow{2}{*}{Conceptualisation du jeu} & \'Ecriture du sc\'enario & Alexandre & Ayemane \\
        \cline{2-4}
        & \'Ecriture des dialogues & Ayemane & Mohamed Aziz \\
        \hline\hline
        % Site Web
        \multirow{3}{*}{Site Web} & Design du site web & Ayemane & Martin \\
        \cline{2-4}
        & Recherche technologie web & Martin & Michaël \\
        \cline{2-4}
        & Development du site web & Martin & Michaël \\
        \hline\hline
        % Texturing
        \multirow{2}{*}{Texturing} & Design des interfaces (UI) & Mohamed Aziz & Ayemane \\
        \cline{2-4}
        & Design personnages / terrains / objet & Ayemane & Alexandre \\
        \hline\hline
        % Intégration des textures
        \multirow{2}{*}{Intégration des Texturing} & Int\'egration des interfaces dans Godot & Mohamed Aziz &  \\
        \cline{2-4}
        & Int\'egration des textures dans Godot & Mohamed Aziz & \\
        \hline\hline
        % Level design
        \multirow{2}{*}{Level design} & Faire un schema de la carte & Alexandre & Mohamed Aziz \\
        \cline{2-4}
        &Level mapping avec les textures & Alexandre & Ayemane \\
        \hline\hline
        % Système d'équipements
        \multirow{3}{*}{Système d'équipements} & Listing items et objets (al\'eatoire) & Alexandre & Martin \\
        \cline{2-4}
        & Gestion al\'eatoire des objets & Alexandre & Martin \\
        \cline{2-4}
        & Equilibrage des objets & Alexandre & Martin \\
        \hline\hline
        % Multijoueur
        & Multijoueur & Mohamed Aziz & Michaël \\
        \hline\hline
        % IA
        \multirow{2}{*}{IA} & Pathfing des mobs & Michaël & Martin \\
        \cline{2-4}
        & Pathfing des mobs & Michaël & Martin \\
        \hline\hline
        % Musique et effet sonores
        & Musique et effet sonores & Alexandre & Mohamed Aziz \\
        \hline


    \end{tabular}
    \caption{R\'epartition des tâches}
    \label{fig:repartition_des_taches}
\end{figure}


% Conceptualisation du jeu
    % Écriture du scenario
    % Écriture des dialogues

% Programmation
    % Proof of Concept / Test en Godot
    % Implémentation / Développement
    % Review / Debug / Publishing

% Site web
    % Design du site
    % Recherche technologie web
    % Développement du site
    
% Texturing
    % Réflexion sur le design globale (dessin)
    % Design des interfaces (UI)
    % Design personnages
    % Design terrains
    % Design objets

% Intégration des textures
    % Intégration des interfaces
    % Intégration des textures personnages
    % Intégration des textures terrains
    % Intégration des textures objets

% Level design
    % Faire un schema de la carte 
    % Level mapping avec les textures

% Système d'équipements
    % Listing items et objets (aléatoire)
    % Boutique (aléatoire) -> PNG et inventaire
    % Armurerie (aléatoire) -> Forgeron

% Mutlijoueur

% IA

% Musique et effet sonores


