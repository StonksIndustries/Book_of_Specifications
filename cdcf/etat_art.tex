Le genre du jeu de rôle (RPG) a une histoire riche et diversifiée, ayant évolué au fil des décennies pour offrir une multitude d'expériences immersives et captivantes aux joueurs du monde entier. 
Considéré comme l'un des genres les plus influents de l'industrie du jeu vidéo, le RPG a connu de nombreux titres à succès, dont certains ont marqué l'histoire du jeu vidéo.
Nous allons étudier certains de ces jeux vidéo notables.
\\

Le tout premier jeu de rôle notable, \textit{Dungeons \& Dragons} souvent abrégé D\&D, a jeté les bases du genre dans les années 1970 en popularisant les mécaniques de jeu de rôle basées sur le papier et le crayon.
En transportant les joueurs dans des mondes imaginaires remplis de quêtes épiques et de personnages fantastiques, D\&D a posé les fondations narratives et les mécanismes de jeu qui ont influencé de nombreux jeux vidéo de rôle ultérieurs.
\\

Parmi les principaux jeux de rôle contemporains qui ont marqué l'industrie, trois titres notables incluent :

\vspace{0.2cm}

\begin{itemize}
    \item \textit{Diablo} : Célébré pour sa réinvention du RPG avec le "point-and-click", ses graphismes réalistes et la qualité des modèles et animations, \textit{Diablo} envoie les joueurs dans un univers prenant et addictif.
    Doté d'une i nterface simple à prendre en main et intégrant de la génération aléatoire de carte, le célèbre jeu a su apporter de la diversité et de la difficulté à ses joueurs. \\

    \item \textit{The Elder Scrolls} : Acclamé pour son vaste monde ouvert et sa liberté de choix, \textit{The Elder Scrolls} offre aux joueurs la possibilité de façonner leur propre destinée dans un univers fantastique riche en aventures et en mystères.
    Parmi ses caractéristiques notables figurent la personnalisation du personnage, les quêtes dynamiques et la possibilité de forger sa propre histoire. \\
    
    \item \textit{Dark Souls} : Renommé pour son gameplay exigeant et sa conception de niveau complexe, \textit{Dark Souls} défie les joueurs avec des défis intenses et des combats stratégiques. Son atmosphère sombre et immersive, combinée à un système de combat gratifiant, a suscité l'admiration des joueurs passionnés de RPG exigeants.

\end{itemize} 

\vspace{0.5cm}

Ces jeux de rôle remarquables se distinguent par leurs mondes captivants, leurs mécaniques de jeu uniques et leurs récits évocateurs, offrant aux joueurs des expériences inoubliables et une immersion profonde dans des univers riches et complexes. 
\\

Étant l’une des licences majeures de Blizzard Entertainment \textit{Diablo}, a grandement façonné les jeux de rôle d'action qui lui ont succédé depuis ses débuts en 1996. 
Plongeant les joueurs dans un monde médiéval fantastique sombre et démoniaque, le jeu se distingue par son gameplay innovant axé sur l'exploration de donjons générés de manière procédurale, la collecte d'objets précieux , combat contre des créatures démoniaques et graphismes de grande qualité pour l’époque, le tout avec la possibilité de jouer en ligne avec d’autres joueurs. 
Avec son action frénétique et son emphase sur la recherche de trésors légendaires, \textit{Diablo} a rapidement gagné en popularité, établissant de nouveaux standards dans le genre des RPG d'action. 
La série, développée par \textit{Blizzard North}, a propulsé le concept de “hack 'n' slash”, où les joueurs peuvent améliorer leurs personnages en accumulant de l'expérience et en équipant des armes et armures de plus en plus puissantes, dans le but. 
La franchise a connu un succès commercial retentissant, avec plus de 2,5 millions d'exemplaires vendus dans le monde, confirmant ainsi sa position emblématique dans l'industrie du jeu vidéo.
\\

La suite The Elder Scrolls, avec son premier jeu publié en 1994 par \textit{Bethesda Softworks}, est la série de jeu de rôle et action la plus populaire du genre des RPG. 
Le dernier jeu de cette série, vendu à plus de 60 million de joueurs, \textit{The Elder Scrolls V} : \textit{Skyrim} est le plus populaire d'entre tous, sortant sur les nouvelles consoles plus de 10 ans après sa sortie en 2011. 
Bien que les jeux principaux sont tous en mode solo, \textit{The Elder Scrolls Online} propose du contenu en multijoueur. 
Une des raisons pour laquelle la série est si populaire est la liberté donnée au joueur, comme par exemple le choix du personnage et de sa race, ou la taille immense de la carte, ayant un total de 12 millions de km\^{2}. 
En effet, cela permet une immersion facile dans la peau de son personnage. 
L'ensemble de ces jeux sont aussi majoritairement basé sur un système de quêtes, qui guident le joueur lors de ses aventures.
\\

Une autre license de jeu de rôle majeure est \textit{Dark Souls}, constituée de trois jeux. 
Le premier jeu de la série est publié en 2011 par FromSoftware, et est suivi de \textit{Dark Souls} II et III en 2014 et 2016. 
Au premier abord, ce jeu n'est pas aussi populaire que ceux cités précédemment, mais il reste indéniablement l'un des jeux vidéo les plus influents. 
Les facteurs qui rendent ce jeu si intéressant sont les suivants. 
Tout d'abord, le gameplay en lui-même se démarque des autres jeux, en se concentrant sur la maîtrise de son personnage ainsi que ses armes et objets, ainsi que les mécaniques de combat. 
Ensuite, le level design pousse à l'exploration quasi constante pour découvrir le moindre passage secret, ainsi que l'histoire du jeu répartie çà et là pour laisser le joueur découvrir et même inventer sa propre histoire 