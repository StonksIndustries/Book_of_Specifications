% Avancement et planification : le cahier des charges doit contenir un planning détaillé d’avancement des tâches par période (temps séparant deux soutenances). 
% Vous le présenterez sous la forme d’un tableau à deux entrées (tâches, soutenances) avec un pourcentage d’avancement.

Afin de suivre l'avancement du projet, nous pouvons lister les t\^aches importantes du projet et estimer leur avancement.
Dans le tableau ci-dessous (cf Figure \ref*{fig:avancement_planification}), nous avons rassemblé les t\^aches et sauvegarder leur avancement à chaque soutenance.
\\




\begin{figure}[H]
    \centering
    \begin{tabular}{|c|c|c|c|}
        \hline
        Tache & Soutenance de méthodologie & 1\up{ème} soutenance & 2\up{ème} soutenance \\
        \hline
        Cahier des charges & \longprogressbar{100}A & & \\
        \hline
        Recherche du concept & \longprogressbar{75}A & & \\
        \hline
        \'Ecriture du scn\'enario & \longprogressbar{0}A & & \\
        \hline
        Site web & \longprogressbar{0}A & & \\
        \hline
        Design de la carte & \longprogressbar{0}A & & \\
        \hline
    \end{tabular}
    \caption{Répartition des tâches}
    \label{fig:avancement_planification}
\end{figure}
