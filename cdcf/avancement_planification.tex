% Avancement et planification : le cahier des charges doit contenir un planning détaillé d’avancement des tâches par période (temps séparant deux soutenances). 
% Vous le présenterez sous la forme d’un tableau à deux entrées (tâches, soutenances) avec un pourcentage d’avancement.

Afin de suivre l'avancement du projet, nous avons listé les tâches importantes du projet et nous en avons fait un planning prévisionnel.
Dans le tableau ci-dessous (cf Figure \ref*{fig:avancement_planification}), on trouve le pourcentage d'avancement de chaque tâche pour chaque soutenance.
Il s'agit donc de synthétiser nos objectifs pour chaque soutenance.
\\

Comme tous les projets, il n'est pas impossible que nous rencontrions des problèmes qui pourraient retarder le développement du jeu.
Avoir un planning fixé dès le début du projet permet de se fixer des objectifs et de pouvoir les atteindre.
Il permet aussi de tester la faisabilité de chaque tâche dans le temps imparti et de pouvoir réagir rapidement en cas de dépassement de la durée allouée.
\\

\begin{figure}[H]
    \centering
    \begin{tabular}{|c|c|c|c|}
        \hline
        Tache & 1\up{\'ere} soutenance & 2\up{ème} soutenance & 3\up{ème} soutenance \\
        \hline
        Site web & + & ++ & +++ \\
        \hline
        Concept du jeu & +++ & +++ & +++ \\
        \hline
        Design des textures & + & ++ & +++ \\
        \hline
        D\'eveloppement du jeu dans godot &  & ++ & +++ \\
        \hline
        Multijoueur &  & ++ & +++ \\
        \hline
        Musique et effets sonores &  & + & +++ \\
        \hline
    \end{tabular} \\
    \vspace*{0.2cm}
    + : Tâche non commencée  /  ++ : Tâche en cours  /  +++ : Tâche terminée\\

    \caption{Répartition des tâches}
    \label{fig:avancement_planification}
\end{figure}





