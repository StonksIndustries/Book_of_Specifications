L'attribution des tâches est primordiale pour le bon déroulement du projet.
Il permet de voir qui est responsable de quelle tâche et de s'assurer que chaque tâche est bien assignée à quelqu'un. 
\\

Chaque tache est attribuée à un membre de l'équipe, il en est le responsable.
On assigne en plus un suppléant à celles-ci, qui pourra prendre le relais en cas d'indisponibilité du responsable. 
Ces rôles sont choisis en fonction des compétences de chacun, mais aussi en fonction de leurs disponibilités. 
En effet une tâche ne peut pas être laissée à l'abandon, elle pourrait mettre en retard tout le projet.
\\

Les tâches sur le long-terme (d'une durée supérieure à une semaine de travail) 
sont listées dans le tableau ci-dessous (cd Figure \ref*{fig:repartition_des_taches}).


\begin{figure}[H]
    \centering
    \begin{tabular}{|c|c|c|c|c|}
        \hline
        Tache & Responsable & Suppléant & Date limite \\
        \hline
        Cahier des charges & Martin & Aziz & \date{25/10/2023} \\
        \hline
        Cahier des charges & Martin & Aziz & \date{25/10/2023} \\
        \hline
    \end{tabular}
    \caption{Répartition des tâches}
    \label{fig:avancement_planification}
\end{figure}
