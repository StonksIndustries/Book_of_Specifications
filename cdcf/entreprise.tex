\companyName est une entreprise qui est spécialisée dans le développement de jeux vidéo attractifs. 
Fondée par un groupe de cinq amis visionnaires et développeurs passionnés, Mohamed Aziz Ben Amor, Ayemane Bouarbi, Alexandre Cölsch, Michaël Museux et Martin Pasquier.
Guidée par l'esprit intrépide de ses fondateurs, l'entreprise s'est donnée pour mission de redéfinir l'expérience ludique grâce à des récits profonds, des visuels époustouflants et un gameplay engageant notamment grâce à leur jeu vidéo, \gameName, qui est en plein développement et cette entreprise a vu le jour dans une salle de l'EPITA, \companyName est née de la volonté de repousser les frontières de l'imaginaire et de l'innovation.
Basée à Paris, l'entreprise est reconnue pour son engagement à repousser les limites de la technologie et du gameplay et de créer des expériences immersives et captivantes pour les joueurs de tous âges. 
L'entreprise est dirigé par Mohamed Aziz Ben Amor, il en est le directeur général, pilote la vision stratégique de l'entreprise, mais aussi par Martin Pasquier qui est le directeur technique de l’entreprise qui lui canalise son perfectionnisme pour garantir que chaque produit soit à la pointe de la technologie.




La \textit{\companyName} est une entreprise spécialisée dans le développement de jeux vidéo depuis maintenant pr\`es de 1 mois.
Créée en 2023 par un groupe de cinq \'etudiants, cette derni\`ere s'est rapidement impos\'ee comme un leader dans l'industrie du jeu vidéo gr\^ace \`a sa passion pour l'innovation et la cr\'eativité. 
Basée à Paris, l'entreprise est reconnue pour son engagement à repousser les limites de la technologie et du gameplay afin de créer des expériences immersives et captivantes pour les joueurs de tous âges.
L'entreprise est dirig\'e par Mohamed Aziz Ben Amor, il en est le directeur général. 
\\

L'ambition autour du jeu \textit{\gameName} est \`a lier \`a son \'equipe d\'evou\'ee \`a cr\'eer des exp\'eriences innovantes, immersives et captivantes. 
L'équipe de développement de \textit{\gameName} est composée de 5 personnes, chacune ayant un rôle spécifique dans le développement du jeu.
\\

\subsubsection*{Gestion \'economique du projet}

Comme toute entreprise, la \textit{\companyName} a aussi des objectifs de rentabilité.
Estimer les co\^uts de développement du jeu, ainsi que les revenus potentiels est donc un enjeu majeur pour l'entreprise, afin d'assurer sa survie et sa croissance.
Le tableau ci-dessous (cf Figure \ref*{fig:couts_de_dev}) a pour objectif d'estimer les co\^uts de développement du jeu.
\\

\begin{figure}[H]
    \centering
    \begin{tabular}{|c|c|c|c|c|}
        \hline
        Cat\'egorie & Nom & Mois de travail & Montant Min & Montant Max \\
        \hline
        \multirow{3}{*}{\'Equipe de d\'eveloppement} & Développeurs C\# & 5 & 450 000 \euro  & 600 000 \euro \\
        \cline{2-5}
        & Game Designer & 5 & 35 000 \euro & 45 000 \euro \\
        \cline{2-5}
        & 2D Artist & 4 & 42 000 \euro & 50 000 \euro \\
        \hline
        Outils et logiciels & Licences logiciels & 5 & 5000 \euro & 7000 \euro \\
        \hline
        \multirow{2}{*}{Communication} & Site web & 2 & 10 000 \euro & 15 000 \euro \\
        \cline{2-5}
        & Marketing & 2 & 10 000 \euro & 15 000 \euro \\
        \hline
         & Serveurs de jeu & 8 & 80 000 \euro & 100 000 \euro \\




    \end{tabular}
    \caption{\'Estimation des co\^uts de développement du jeu}
    \label{fig:couts_de_dev}
\end{figure}




