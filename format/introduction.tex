% En correction

Dans ce vaste monde qu’est le jeu vidéo, cinq étudiants débarquèrent de nulle part avec en tête une idée ambitieuse, créer de toutes pièces un jeu vidéo.
Dans une funeste atmosphère que représente une sombre salle machine de l’EPITA, très tard sous une nuit de pleine lune, fut créée la \textit{\companyName}.
Dans un but commun de devenir une des entreprises les plus prestigieuses dans ce monde, le premier jeu de la \textit{\companyName}, nommé \textit{\gameName}, vit le jour.
\\

\textit{\gameName} est avant tout un jeu vidéo de rôle se déroulant dans le monde médiéval fantastique de \textit{Azerith}, qui est contrôlé par un régime impérial.
Le joueur aura le choix d’incarner plusieurs classes d’humains dans une faction indépendantiste et aura comme but de combattre l’empire en place et de le renverser, dans un monde où de nombreuses créatures malveillantes errent sans but dans la nuit.
\\

L'initiative autour de \textit{\gameName} découle de la demande du marché ainsi que de la validation du premier semestre, l’objectif principal est de développer un jeu captivant avec des graphismes de haute qualité pour assurer un engagement maximal des joueurs.
Sont impliqués dans ce projet les équipes de développement, d’art et de scénario, ainsi que les différents enseignants et jurys d’EPITA.
Ce cahier définit les exigences requises à la conception de ce jeu ainsi que les différentes échéances à venir.
Ces exigences comprennent des graphismes avancés, une jouabilité équilibrée sous Windows et le choix d’un moteur de jeu parmi la liste présentée par EPITA.
Par ailleurs, \textit{\companyName} s’engage à respecter les droits d’auteur pour les différentes ressources qui vont être utilisées.
La validation de ce cahier des charges sera assurée par les enseignants d’EPITA, et le suivi de ce jeu sera effectué lors de deux soutenances en 2024.
\\

Tout d’abord, ce document présentera le Cahier des Charges Fonctionnel (CdCF), qui se base sur les besoins fonctionnels du projet et la manière dont il sera traité sur le temps imparti.
Il se divise en plusieurs grandes catégories que sont les présentations de l’origine et du type du projet, des buts et intérêts ou encore des inspirations pour le projet, ainsi que la présentation de l’entreprise, la présentation de ses membres, la répartition des tâches et enfin un planning d’avancement des tâches.
Le projet aura à la fois un aspect fonctionnel, technologique, méthodologique et opérationnel, qui sera la structure même du CdCF.
\\

Enfin, ce dossier mettra en valeur le Cahier des Charges Technique (CdCT), qui détaille quant à lui les contraintes et les spécificités techniques nécessaires pour répondre aux besoins exprimés dans le cahier des charges fonctionnel.
Autrement dit, le CdCT détermine les moyens pour arriver à un résultat espéré sur le projet.
On y liste donc le choix des technologies, les exigences de performance, les critères d'acceptation, les exigences de compatibilité ou encore les caractéristiques techniques à respecter.

