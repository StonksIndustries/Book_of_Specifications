Ce projet a pour but de créer un jeu vidéo de rôle mutlijoueur avec une histoire passionnante.
Le Development de \textit{\gameName} va nous permettre d’acquérir de nouvelles compétences en C\#, en développement web, en gestion de projet et en travail d'équipe.
Notre objectif est de créer une expérience pour les joueurs, un jeu qui les rassemble et qui leur permet de s'amuser ensemble.
Explorer le monde du jeu doit être soit un plaisir pour le joueur, soit une expérience immersive et intrigante.

Dans ce cahier des charges, nous avons abordés tous les points qui définissent notre projet.
Nous avons tous des histoires différentes, des compétences différentes et des objectifs différents mais nous avons tous un point commun : nous voulons créer \textit{\gameName}.

La gestion d'un groupe de travail et projet d'équipe est une compétence importante dans le monde du travail.
L'acquérir dès la première année de notre formation est donc un bon point pour notre avenir professionnel.

Bien que des difficultés soient à prévoir, nous sommes confiants quant à la réussite de ce projet. 
Ce sera une expérience enrichissante pour nous tous et formatrice des attentes du monde professionnel.
Ce projet est aussi important dans le cadre de notre formation d'ingénieur à l'EPITA.
En effet, il constitue une première expérience de travail en équipe sur un projet de grande envergure, avec des contraintes de temps et de qualité de travail.

Il s'agit pour nous d'explorer par nous même les aspects complexes du développement de jeux vidéo, et de nous confronter aux problématiques de ce métier pendant notre formation.

