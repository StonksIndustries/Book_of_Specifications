\documentclass[11pt]{article}

% French
\usepackage[utf8]{inputenc}
\usepackage[T1]{fontenc}
\usepackage[french]{babel}

\usepackage{float}
\usepackage{pgfgantt}
\usepackage{geometry}

\geometry{margin=2.5cm}

\title{
    Cahier des Charges projet \\
    \textbf{Lands of Stonkseria}
    \vspace{7cm}
}
\author{
    Mohamed Aziz Ben Amor (Chief Executive Officer) \\
    \texttt{mohamed-aziz.ben-amor@epita.fr}
    \vspace{0.5cm}\and
    Martin Pasquier (Technical director) \\
    \texttt{martin.pasquier@example.com}
    \vspace{0.5cm}\and
    Alexandre Cölsch (Game Designer) \\
    \texttt{alexandre.colsch@epita.fr}
    \vspace{0.5cm}\and
    Ayemane Bouarbi (Concept Artist) \\
    \texttt{ayemane.bouarbi@epita.fr}
    \vspace{0.5cm}\and
    Michaël Museux (Software Engineer) \\
    \texttt{martin.pasquier@example.com}
    \vspace{1.5cm}\and
    \textbf{EPITA} \\
    \textbf{Stonks Industries} \\\\
}

\date{\today}

\begin{document}

\begin{titlepage}
    \maketitle
    \thispagestyle{empty} % Remove page number from title page
\end{titlepage}

\newpage
\tableofcontents

\newpage
\section{Introduction}


\newpage
\section{Cahier des Charges Fonctionnel}

\subsection{Origine et Nature du Projet}

\subsection{Objet de l'\'Etude}

\subsection{\'Etat de l'Art}

\subsection{Notre Entreprise}

\subsection{Notre \'Equipe}

\subsection{La Répartition des Tâches}

L'attribution des tâches est primordiale pour le bon déroulement du projet.
Il permet de voir qui est responsable de quelle tâche et de s'assurer que chaque tâche est bien assignée à quelqu'un. 

Chaque tache est attribuée à un membre de l'équipe, il en est le responsable.
On assigne en plus un suppléant à celles-ci, qui pourra prendre le relais en cas d'indisponibilité du responsable. 
Ces rôles sont choisit en fonction des compétences de chacun, mais aussi en fonction de leurs disponibilités. 
En effet une tache ne peux pas être laissé à l'abandon, elle pourrait mettre en retard tout le projet.

Les tâches sur le long-terme (supérieur à une semaine de travail) sont listés dans le tableau ci-dessous (cd Figure \ref*{fig:repartition_des_taches}).

\begin{figure}[H]
    \centering
    \begin{tabular}{|c|c|c|c|c|}
        \hline
        Tache & Responsable & Suppléant & Date limite \\
        \hline
        Cahier des charges & Martin & Aziz & \date{25/10/2023} \\
        \hline
        Cahier des charges & Martin & Aziz & \date{25/10/2023} \\
        \hline
    \end{tabular}
    \caption{Répartition des tâches}
    \label{fig:repartition_des_taches}
\end{figure}


\subsection{Avancement et Planification}

Un Diagramme de Gantt est disponible ci-dessous (cf. Figure \ref{fig:diagramme_de_gantt}) pour illustrer la répartition des tâches dans le temps.
Il permet d'avoir une vision globale du projet et de son avancement. C'est un bon outils pour prévoir les retards et les anticiper.

\begin{figure}[H]
    \centering
    \begin{ganttchart}[
      hgrid,
      vgrid,
      today=2,
      group left shift=0,
      group right shift=0,
      group peaks tip position=0,
      group height=.3,
      y unit chart=0.7cm,
    ]{1}{18}
      \gantttitlelist{"Oct.","Nov.","Dec.","Janv.","Fevr","Mars","Avril","Mai","Juin"}{2} \\
  
      \ganttgroup{Concept}{1}{4} \\
      \ganttbar[progress=75]{Concept Establishment}{1}{2} \\
      \ganttbar[progress=0]{Scenario Writing}{3}{4} \\  
      \ganttnewline
  
      \ganttgroup{Design}{5}{10} \\
      \ganttbar[progress=0]{Texture Design}{5}{10} \\
      \ganttbar[progress=0]{Texture Review}{8}{8} \\
      \ganttbar[progress=0]{Integration}{9}{10} \\
      \ganttnewline
  
      \ganttgroup{Programming}{9}{14} \\
      \ganttbar[progress=0]{Proof of Concept / Test}{9}{14} \\
      \ganttbar[progress=0]{Implementation / Development}{11}{13} \\
      \ganttbar[progress=0]{Review / Debug / Publishing}{14}{14} \\
      \ganttnewline
    
      \ganttgroup{Website}{5}{8} \\
      \ganttbar[progress=0]{Design on figma}{5}{6} \\
      \ganttbar[progress=0]{Web Development}{6}{8} \\
      \ganttbar[progress=0]{Review / Debug / Publishing}{8}{8} \\
      \ganttnewline
  
    \end{ganttchart}
    \caption{Diagramme de Gantt}
    \label{fig:diagramme_de_gantt}
  \end{figure}


\newpage
\section{Cahier des Charges Technique}


\newpage
\section{Conclusion}


\end{document}